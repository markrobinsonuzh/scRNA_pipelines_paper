%% BioMed_Central_Tex_Template_v1.06
%%                                      %
%  bmc_article.tex            ver: 1.06 %
%                                       %

%%IMPORTANT: do not delete the first line of this template
%%It must be present to enable the BMC Submission system to
%%recognise this template!!

%%%%%%%%%%%%%%%%%%%%%%%%%%%%%%%%%%%%%%%%%
%%                                     %%
%%  LaTeX template for BioMed Central  %%
%%     journal article submissions     %%
%%                                     %%
%%          <8 June 2012>              %%
%%                                     %%
%%                                     %%
%%%%%%%%%%%%%%%%%%%%%%%%%%%%%%%%%%%%%%%%%


%%%%%%%%%%%%%%%%%%%%%%%%%%%%%%%%%%%%%%%%%%%%%%%%%%%%%%%%%%%%%%%%%%%%%
%%                                                                 %%
%% For instructions on how to fill out this Tex template           %%
%% document please refer to Readme.html and the instructions for   %%
%% authors page on the biomed central website                      %%
%% http://www.biomedcentral.com/info/authors/                      %%
%%                                                                 %%
%% Please do not use \input{...} to include other tex files.       %%
%% Submit your LaTeX manuscript as one .tex document.              %%
%%                                                                 %%
%% All additional figures and files should be attached             %%
%% separately and not embedded in the \TeX\ document itself.       %%
%%                                                                 %%
%% BioMed Central currently use the MikTex distribution of         %%
%% TeX for Windows) of TeX and LaTeX.  This is available from      %%
%% http://www.miktex.org                                           %%
%%                                                                 %%
%%%%%%%%%%%%%%%%%%%%%%%%%%%%%%%%%%%%%%%%%%%%%%%%%%%%%%%%%%%%%%%%%%%%%

%%% additional documentclass options:
%  [doublespacing]
%  [linenumbers]   - put the line numbers on margins

%%% loading packages, author definitions

%\documentclass[twocolumn]{bmcart}% uncomment this for twocolumn layout and comment line below
\documentclass{bmcart}

%%% Load packages
%\usepackage{amsthm,amsmath}
%\RequirePackage{natbib}
%\RequirePackage[authoryear]{natbib}% uncomment this for author-year bibliography
%\RequirePackage{hyperref}
\usepackage[utf8]{inputenc} %unicode support
%\usepackage[applemac]{inputenc} %applemac support if unicode package fails
%\usepackage[latin1]{inputenc} %UNIX support if unicode package fails
\usepackage[pdf]{graphviz}

%%%%%%%%%%%%%%%%%%%%%%%%%%%%%%%%%%%%%%%%%%%%%%%%%
%%                                             %%
%%  If you wish to display your graphics for   %%
%%  your own use using includegraphic or       %%
%%  includegraphics, then comment out the      %%
%%  following two lines of code.               %%
%%  NB: These line *must* be included when     %%
%%  submitting to BMC.                         %%
%%  All figure files must be submitted as      %%
%%  separate graphics through the BMC          %%
%%  submission process, not included in the    %%
%%  submitted article.                         %%
%%                                             %%
%%%%%%%%%%%%%%%%%%%%%%%%%%%%%%%%%%%%%%%%%%%%%%%%%


\def\includegraphic{}
\def\includegraphics{}



%%% Put your definitions there:
\startlocaldefs
\endlocaldefs


%%% Begin ...
\begin{document}

%%% Start of article front matter
\begin{frontmatter}

\begin{fmbox}
\dochead{Research}

%%%%%%%%%%%%%%%%%%%%%%%%%%%%%%%%%%%%%%%%%%%%%%
%%                                          %%
%% Enter the title of your article here     %%
%%                                          %%
%%%%%%%%%%%%%%%%%%%%%%%%%%%%%%%%%%%%%%%%%%%%%%

\title{A sample article title}

%%%%%%%%%%%%%%%%%%%%%%%%%%%%%%%%%%%%%%%%%%%%%%
%%                                          %%
%% Enter the authors here                   %%
%%                                          %%
%% Specify information, if available,       %%
%% in the form:                             %%
%%   <key>={<id1>,<id2>}                    %%
%%   <key>=                                 %%
%% Comment or delete the keys which are     %%
%% not used. Repeat \author command as much %%
%% as required.                             %%
%%                                          %%
%%%%%%%%%%%%%%%%%%%%%%%%%%%%%%%%%%%%%%%%%%%%%%

\author[
   addressref={aff1,aff2},                   % id's of addresses, e.g. {aff1,aff2}
   corref={aff1},                       % id of corresponding address, if any
   email={pierre-luc.germain@hest.ethz.ch}   % email address
]{\inits{PLG}\fnm{Pierre-Luc} \snm{Germain}}
\author[
   addressref={aff1},
   corref={aff1},                       % id of corresponding address, if any
   email={mark.robinson@imls.uzh.ch}
]{\inits{MR}\fnm{Mark} \snm{Robinson}}

%%%%%%%%%%%%%%%%%%%%%%%%%%%%%%%%%%%%%%%%%%%%%%
%%                                          %%
%% Enter the authors' addresses here        %%
%%                                          %%
%% Repeat \address commands as much as      %%
%% required.                                %%
%%                                          %%
%%%%%%%%%%%%%%%%%%%%%%%%%%%%%%%%%%%%%%%%%%%%%%

\address[id=aff1]{
  \orgname{Statistical Bioinformatics, IMLS, University of Z\"{u}rich}, % university, etc
  \street{Winterthurerstrasse 190},
  \postcode{8057}
  \city{Z\"{u}erich},
  \cny{Switzerland}
}
\address[id=aff2]{
  \orgname{D-HEST Institute for Neurosciences, ETH Z\"{u}rich},
  \street{Winterthurerstrasse 190},
  \postcode{8057}
  \city{Z\"{u}erich},
  \cny{Switzerland}
}

%%%%%%%%%%%%%%%%%%%%%%%%%%%%%%%%%%%%%%%%%%%%%%

\end{fmbox}% comment this for two column layout

%%%%%%%%%%%%%%%%%%%%%%%%%%%%%%%%%%%%%%%%%%%%%%
%%                                          %%
%% The Abstract begins here                 %%
%%                                          %%
%% Please refer to the Instructions for     %%
%% authors on http://www.biomedcentral.com  %%
%% and include the section headings         %%
%% accordingly for your article type.       %%
%%                                          %%
%%%%%%%%%%%%%%%%%%%%%%%%%%%%%%%%%%%%%%%%%%%%%%

\begin{abstractbox}

\begin{abstract} % abstract
\parttitle{Background} %if any
Text for this section.
\parttitle{Results} %if any
Text for this section.
\parttitle{Conclusions}
We offer 
\end{abstract}

%%%%%%%%%%%%%%%%%%%%%%%%%%%%%%%%%%%%%%%%%%%%%%
%%                                          %%
%% The keywords begin here                  %%
%%                                          %%
%% Put each keyword in separate \kwd{}.     %%
%%                                          %%
%%%%%%%%%%%%%%%%%%%%%%%%%%%%%%%%%%%%%%%%%%%%%%

\begin{keyword}
\kwd{single-cell RNAseq}
\kwd{pipeline}
\kwd{clustering}
\kwd{filtering}
\kwd{benchmark}
\end{keyword}

% MSC classifications codes, if any
%\begin{keyword}[class=AMS]
%\kwd[Primary ]{}
%\kwd{}
%\kwd[; secondary ]{}
%\end{keyword}

\end{abstractbox}
%
%\end{fmbox}% uncomment this for twcolumn layout

\end{frontmatter}

%%%%%%%%%%%%%%%%%%%%%%%%%%%%%%%%%%%%%%%%%%%%%%
%%                                          %%
%% The Main Body begins here                %%
%%                                          %%
%% Please refer to the instructions for     %%
%% authors on:                              %%
%% http://www.biomedcentral.com/info/authors%%
%% and include the section headings         %%
%% accordingly for your article type.       %%
%%                                          %%
%% See the Results and Discussion section   %%
%% for details on how to create sub-sections%%
%%                                          %%
%% use \cite{...} to cite references        %%
%%  \cite{koon} and                         %%
%%  \cite{oreg,khar,zvai,xjon,schn,pond}    %%
%%  \nocite{smith,marg,hunn,advi,koha,mouse}%%
%%                                          %%
%%%%%%%%%%%%%%%%%%%%%%%%%%%%%%%%%%%%%%%%%%%%%%

%%%%%%%%%%%%%%%%%%%%%%%%% start of article main body
% <put your article body there>

%%%%%%%%%%%%%%%%
%% Background %%
%%
\section*{Background}

Single-cell RNA-sequencing (scRNAseq) and the set of attached analysis methods are evolving fast, and while a number of good comparison/benchmark studies can inform analytical decisions [REFs], they need constant updating and often leave open many details of an analysis. Here, we go in more detail on the various steps of analysis leading from an initial count matrix to a cluster assignment, which are critical steps in wide range of applications. Weharness datasets of known cell composition [TABLE 1] to investigate the impact of various parameters and variations around a core pipeline on various multilevel evaluation metrics.

Although we use some datasets based on other protocols, we focus especially on the 10X datasets and on methods that do not require exogenous control RNA, i.e. spike-ins (see Table 1 and Supplementary Figure 1 for a description of the datasets). As Seurat [REF] was previously shown to outperform other clustering approaches [REFs], we used the Seurat pipeline as the starting framework to perform an integrated investigation of 1) doublet identification, 2) cell filtering, 3) normalization, 4) feature selection, 5) dimensional reduction, 6) clustering. We compare not only competing approaches, but also more fine-grained parameter variations on common methods. Importantly, the success of methods at a certain analytical step might be dependent on choices at other steps. Therefore, instead of evaluating each step in isolation, we developed a general R framework for evaluating nested variations on a pipeline, and suggest a multilevel panel of metrics. Finally, we evaluate several recent methods and provide concrete recommendations.

\section*{Results}

\subsection*{A flexible framework for pipeline evaluation}

We defined a pipeline, in R, as a list of functions executed consecutively on the output of the previous one...  The basic pipeline used through most of this paper is described in Figure XXX. A number of metrics can be considered at each step which can be used to evaluate this and the previous steps. 

\subsection*{Doublet detection}

Doublets, defined as two cells sequenced under the same cellular barcode (for instance as a result of being captured in the same droplet), are fairly frequent in scRNAseq datasets, with estimates ranging from 1 to 10\% depending on the platform and cell concentration used \citep{bloom_estimating_2018,kang_multiplexed_2018}. While doublets of the same cell type are relatively innocuous for most downstream applications due to their conservation of the relative expression between genes, doublets formed from different cell types or states are likely to be misclassified and could potentially distort downstream analysis. In some cases, doublets can be identified through their unusually high number of reads and detected features, but this is not always the case (Supplementary Figure 2). A number of methods were developed to identify doublets, in most cases by comparing each cell to artificially-created doublets [REFs]. We therefore first evaluated the capacity of these methods to detect doublets using the two 10x datasets produced by Tian et al. (\citey{tian_mixology_2018}) with cells of different genetic identity, and for which therefore SNP information can be used as ground truth. We tested DoubletFinder [REF] and scran's doubletCells [REF], both of which use similarity to artificial doublets, and scds [REF], which relies on a combination of coexpression and binary classification. DoubletFinder integrates a thresholding based on the proportion of expected doublets, while scran and scds return scores which must be manually thresholded. In these cases, we ensured that the right number of cells would be called doublets.

In addition to these methods, we reasoned that an approach such as DoubletFinder could be considerably sped up by being applied directly on counts and by using a pre-clustering to create neotypic doublets more efficiently. We therefore developed a simple and fast package implementing this for doublet detection, with the added perk of accounting for uncertainty in the expected doublet rate and using meta-cells from the clusters to even include triplets (see methods).

While most methods accurately identified the doublets in the 3 cell lines dataset (mixology10x3cl), the 5 lines dataset (mixology10x5cl) proved more difficult (Figure 1B). Our method achieved the best accuracy while running in a very reasonable time (Figure 1). Across datasets, cells called as doublets tended to be classified in the wrong cluster more often than other cells (Figure 1D). We therefore tested whether this method also improved the accuracy of the clustering across all benchmark datasets, and found that it consistently did (Figure 2).

\subsection*{Filtering: excluded more cells is not necessarily good}

Beyond doublets, a dataset might include low-quality cells whose elimination would reduce noise. This has for instance been demonstrated for cells that have a high content of mitochondrial reads, often as a result of cell degradation [REF]. A common practice is to exclude cells that differ considerably from most other cells on the basis of some such properties, for instance through the `isOutlier` function of `scater` which measures, for a given control property, the number of median absolute deviation of each cell from the mean of all cells. Supplementary Figure 1 shows the distributions of some of the typical cell properties commonly used. Of note, these properties tend to be correlated, but are not always: for instance, while a high proportion of mitochondrial reads is often correlated with a high proportion of the counts in the top features, there can also be other reasons for an over-representation of highly-expressed features (Supplementary Figure 3). In addition, in our experience 10X datasets also exhibit a very tight correlation between the total counts and the total features even across very different cell types (Supplementary Figure 4). We therefore also measure the ratio between the two, and treat cells strongly departing from this trend with suspicion.

Reasoning that the cells we wish to eliminate are cells that would be misclassified, we measured the rate of misclassification of each cell in each dataset across a variety of clustering pipelines, correcting for the median misclassification rate of the subpopulation, and then evaluated what properties could be predictive of misclassification (Supplementary Figures 5-7). We could not identify any property or simple combination thereof which would be consistently predictive of misclassification; the only feature that consistently stood out across multiple datasets (the Zheng datasets) was that cells with a very high read counts have a higher chance of being misclassified.

We next investigated the impact of filtering according to various criteria on the accuracy of the clustering. We tested each aforementioned cell property, alone and in combinations, filtering above and/or below a certain threshold defined by different numbers of median absolute deviations. Reasoning that a bad cell would be bad in several distributions, we also included the optional rule that a cell should be excluded if it is an outlier in at least 2 or 3 distributions.

A first observation was that very stringent filtering (removing up to 20\% of the cells) was not generally associated with an increase in accuracy of the clustering (Supplementary Figure 8A). Moreover, stringent filtering tended to be biased towards the exclusion of specific subpopulations (Supplementary Figure 8B). Indeed, applying the same filtering criteria on individual clusters of cells (identified through scran's `quickCluster` method) resulted in nearly no cell being filtered out, which suggests that filtering on the global population tends to discard cells of subpopulations with more extreme properties, rather than `bad` cells.

We next investigated the contribution of various criteria individually (Supplementary Figure 9). Interestingly, regression analysis suggests that increasing the number of distributions on which a cell must deviate in order to be excluded tended to be positively associated with ARI, while increasing the number of median absolute deviations required to be an outlier as associated with a decrease in ARI. We therefore suggest excluding cells that are outliers on more than 1 distributions. The impact of filtering on the basis of total counts or feature counts was not always robust, nor always positive, and we could detect no significant impact of filtering on the basis of the proportion of mitochondrial reads - however it must be said that the benchmark datasets count very few cells with a very high proportion of mitochondrial reads (Supplementary Figure 1). We therefore suggest to keep this criterion, but in a relatively lenient fashion, and not solely based on the median absolute deviation of the distribution. Filtering solely based on deviations from the distribution might be inadequate when the distribution is tight; in our experience, the proportion of mitochondrial counts in otherwise `healthy` cells can vary up to 8-12\% depending on the tissue. We therefore recommend using a combination of fixed and deviation based threshold, e.g. x $>$ 0.08 and x $>$ 2.5*MADs.

On the basis of these observations, we defined three sets of criteria as lenient, default and stringent filtering (see methods), and investigated them further. 

\subsection*{Normalization and scaling}





\subsection*{Sub-heading for section}
Text for this sub-heading \ldots
\subsubsection*{Sub-sub heading for section}
Text for this sub-sub-heading \ldots
\paragraph*{Sub-sub-sub heading for section}
Text for this sub-sub-sub-heading \ldots
In this section we examine the growth rate of the mean of $Z_0$, $Z_1$ and $Z_2$. In
addition, we examine a common modeling assumption and note the
importance of considering the tails of the extinction time $T_x$ in
studies of escape dynamics.
We will first consider the expected resistant population at $vT_x$ for
some $v>0$, (and temporarily assume $\alpha=0$)

Thus we observe that this expected value is finite for all $v>0$ (also see \cite{koon,khar,zvai,xjon,marg}).
%\nocite{oreg,schn,pond,smith,marg,hunn,advi,koha,mouse}

\section*{Discussion}

For research articles this section should discuss the implications of the findings in context of existing research and highlight limitations of the study. For study protocols and methodology manuscripts this section should include a discussion of any practical or operational issues involved in performing the study and any issues not covered in other sections.

\section*{Conclusions}

This should state clearly the main conclusions and provide an explanation of the importance and relevance of the study to the field.

\section*{Methods}


%%%%%%%%%%%%%%%%%%%%%%%%%%%%%%%%%%%%%%%%%%%%%%
%%                                          %%
%% Backmatter begins here                   %%
%%                                          %%
%%%%%%%%%%%%%%%%%%%%%%%%%%%%%%%%%%%%%%%%%%%%%%

\begin{backmatter}

\section*{Competing interests}
  The authors declare that they have no competing interests.

\section*{Author's contributions}
    Text for this section \ldots

\section*{Acknowledgements}
  Text for this section \ldots
%%%%%%%%%%%%%%%%%%%%%%%%%%%%%%%%%%%%%%%%%%%%%%%%%%%%%%%%%%%%%
%%                  The Bibliography                       %%
%%                                                         %%
%%  Bmc_mathpys.bst  will be used to                       %%
%%  create a .BBL file for submission.                     %%
%%  After submission of the .TEX file,                     %%
%%  you will be prompted to submit your .BBL file.         %%
%%                                                         %%
%%                                                         %%
%%  Note that the displayed Bibliography will not          %%
%%  necessarily be rendered by Latex exactly as specified  %%
%%  in the online Instructions for Authors.                %%
%%                                                         %%
%%%%%%%%%%%%%%%%%%%%%%%%%%%%%%%%%%%%%%%%%%%%%%%%%%%%%%%%%%%%%

% if your bibliography is in bibtex format, use those commands:
\bibliographystyle{bmc-mathphys} % Style BST file (bmc-mathphys, vancouver, spbasic).
\bibliography{bmc_article}      % Bibliography file (usually '*.bib' )
% for author-year bibliography (bmc-mathphys or spbasic)
% a) write to bib file (bmc-mathphys only)
% @settings{label, options="nameyear"}
% b) uncomment next line
%\nocite{label}

% or include bibliography directly:
% \begin{thebibliography}
% \bibitem{b1}
% \end{thebibliography}

%%%%%%%%%%%%%%%%%%%%%%%%%%%%%%%%%%%
%%                               %%
%% Figures                       %%
%%                               %%
%% NB: this is for captions and  %%
%% Titles. All graphics must be  %%
%% submitted separately and NOT  %%
%% included in the Tex document  %%
%%                               %%
%%%%%%%%%%%%%%%%%%%%%%%%%%%%%%%%%%%

%%
%% Do not use \listoffigures as most will included as separate files

\section*{Figures}
  \begin{figure}[h!]
  \caption{\csentence{Sample figure title.}
      A short description of the figure content
      should go here.}
      \end{figure}

\begin{figure}[h!]
  \caption{\csentence{Sample figure title.}
      Figure legend text.}
      \end{figure}

%%%%%%%%%%%%%%%%%%%%%%%%%%%%%%%%%%%
%%                               %%
%% Tables                        %%
%%                               %%
%%%%%%%%%%%%%%%%%%%%%%%%%%%%%%%%%%%

%% Use of \listoftables is discouraged.
%%
\section*{Tables}
\begin{table}[h!]
\caption{Sample table title. This is where the description of the table should go.}
      \begin{tabular}{cccc}
        \hline
           & B1  &B2   & B3\\ \hline
        A1 & 0.1 & 0.2 & 0.3\\
        A2 & ... & ..  & .\\
        A3 & ..  & .   & .\\ \hline
      \end{tabular}
\end{table}

%%%%%%%%%%%%%%%%%%%%%%%%%%%%%%%%%%%
%%                               %%
%% Additional Files              %%
%%                               %%
%%%%%%%%%%%%%%%%%%%%%%%%%%%%%%%%%%%

\section*{Additional Files}
  \subsection*{Additional file 1 --- Sample additional file title}
    Additional file descriptions text (including details of how to
    view the file, if it is in a non-standard format or the file extension).  This might
    refer to a multi-page table or a figure.

  \subsection*{Additional file 2 --- Sample additional file title}
    Additional file descriptions text.


\end{backmatter}
\end{document}
